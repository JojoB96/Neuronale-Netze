% !TeX encoding=UTF-8
% !TeX spellcheck=de_DE_frami
 
\documentclass[a4paper,12pt,titlepage]{article}


\usepackage{amsthm}

% \include{Dateiname}
% -> faengt neue Seite mit dem Inhalt
%    aus Dateiname.tex an
% \input{Dateiname}
% -> fuegt den Inhalt aus Dateiname.tex
%    direkt ein
\usepackage[utf8]{inputenc}


\begin{document}
\begin{centering}%
\textbf{\Large Reinforcement Learning am Beispiel Mancala}\\[0.5cm]
\textbf{Johanna Beier} und \textbf{Viktor Kosin}\\
Leibniz Universität Hannover\\
Institut für Angewandte Mathematik\\[1cm]
\end{centering}
\textbf{Abstract} \\ \ \\
This presentation deals with the training of a neural network to learn playing the two player game mancala, especially the german version \textit{Bohnenspiel}. As a trainingmethod we use reinforcement learning. We trained the agent by having it play serveral times against itself. The agent gets a small reward for choosing states in which it catches beans and a bigger reward if the game is won. While training we allow a small possibility to do random steps. The trained net is tested by counting how many games are won while playing against a random player or ???
ergebnisse: ??? gut? Reinforcement geeignet?
\end{document}
